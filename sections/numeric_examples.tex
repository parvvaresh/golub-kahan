\section{مثال عددی از الگوریتم بیدیاگونالیزاسیون گولوب-کاهان}

فرض کنید ماتریس \( A \) به صورت زیر باشد:
\[
A = \begin{bmatrix}
1 & 2 & 3 \\
4 & 5 & 6 \\
7 & 8 & 9
\end{bmatrix}
\]

\subsection*{گام اول: بازتاب هوسهولدر از چپ \lr{(i = 1)}}
\begin{align*}
B &= \begin{bmatrix}
1 & 2 & 3 \\
4 & 5 & 6 \\
7 & 8 & 9
\end{bmatrix} \\
x &= \begin{bmatrix}
1 \\
4 \\
7
\end{bmatrix} \\
v &= \begin{bmatrix}
0.75 \\
0.33 \\
0.57
\end{bmatrix} \\
H &= I - 2 \frac{vv^T}{v^T v} = \begin{bmatrix}
-0.12 & -0.49 & -0.86 \\
-0.49 &  0.78 & -0.38 \\
-0.86 & -0.38 &  0.34
\end{bmatrix} \\
\text{ضرب } H \cdot B &= \begin{bmatrix}
-0.12 & -0.49 & -0.86 \\
-0.49 &  0.78 & -0.38 \\
-0.86 & -0.38 &  0.34
\end{bmatrix} \cdot \begin{bmatrix}
1 & 2 & 3 \\
4 & 5 & 6 \\
7 & 8 & 9
\end{bmatrix} = \begin{bmatrix}
 -8.12 &  -9.6 &  -11.08 \\
  0 &  -0.09 &  -0.17 \\
  0 &  -0.9 &  -1.8
\end{bmatrix} \\
\text{بروزرسانی } U &= I \cdot H = \begin{bmatrix}
-0.12 & -0.49 & -0.86 \\
-0.49 &  0.78 & -0.38 \\
-0.86 & -0.38 &  0.34
\end{bmatrix}
\end{align*}

\subsection*{گام دوم: بازتاب هوسهولدر از راست \lr{(i = 0)}}
\begin{align*}
B &= \begin{bmatrix}
 -8.12 &  -9.6 &  -11.08 \\
  0 &  -0.09 &  -0.17 \\
  0 &  -0.9 &  -1.8
\end{bmatrix} \\
x &= \begin{bmatrix}
 -9.6 \\
 -11.08
\end{bmatrix} \\
v &= \begin{bmatrix}
-0.91 \\
-0.42
\end{bmatrix} \\
H &= I - 2 \frac{vv^T}{v^T v} = \begin{bmatrix}
-0.65 & -0.76 \\
-0.76 &  0.65
\end{bmatrix} \\
\text{ضرب } B \cdot H &= \begin{bmatrix}
 -9.6011363 & -11.07823419 \\
 -0.08596557 & -0.17193114 \\
 -0.90043975 & -1.8008795
\end{bmatrix} \cdot \begin{bmatrix}
-0.65 & -0.76 \\
-0.76 &  0.65
\end{bmatrix} = \begin{bmatrix}
-8.12 & 14.66 & -0 \\
 0 & 0.19 & -0.05 \\
 0 & 1.95 & -0.5
\end{bmatrix} \\
\text{بروزرسانی } V &= \begin{bmatrix}
 1 & 0 & 0 \\
 0 & -0.65 & -0.76 \\
 0 & -0.76 &  0.65
\end{bmatrix}
\end{align*}

\subsection*{گام سوم: بازتاب هوسهولدر از چپ \lr{(i = 1)}}
\begin{align*}
B &= \begin{bmatrix}
-8.12 & 14.66 & -0 \\
 0 & 0.19 & -0.05 \\
 0 & 1.95 & -0.5
\end{bmatrix} \\
x &= \begin{bmatrix}
0.19 \\
1.95
\end{bmatrix} \\
v &= \begin{bmatrix}
0.74 \\
0.67
\end{bmatrix} \\
H &= I - 2 \frac{vv^T}{v^T v} = \begin{bmatrix}
-0.1 & -1 \\
-1 & 0.1
\end{bmatrix} \\
\text{ضرب } H \cdot B &= \begin{bmatrix}
-0.1 & -1 \\
-1 & 0.1
\end{bmatrix} \cdot \begin{bmatrix}
 0 & 0.19 & -0.05 \\
 0 & 1.95 & -0.5
\end{bmatrix} = \begin{bmatrix}
-8.12 & 14.66 & -0 \\
 -0 & -1.96 & 0.5 \\
 -0 & -0 & 0
\end{bmatrix} \\
\text{بروزرسانی } U &= \begin{bmatrix}
-0.49 & -0.86 \\
 0.78 & -0.38 \\
-0.38 & 0.34
\end{bmatrix} \cdot \begin{bmatrix}
-0.1 & -1 \\
-1 & 0.1
\end{bmatrix} = \begin{bmatrix}
-0.12 & 0.9 & 0.41 \\
-0.49 & 0.3 & -0.82 \\
-0.86 & -0.3 & 0.41
\end{bmatrix}
\end{align*}

\subsection*{گام چهارم: بازتاب هوسهولدر از راست \lr{(i = 1)}}
\begin{align*}
B &= \begin{bmatrix}
-8.12 & 14.66 & -0 \\
 -0 & -1.96 & 0.5 \\
 -0 & -0 & 0
\end{bmatrix} \\
x &= \begin{bmatrix}
0.5
\end{bmatrix} \\
v &= \begin{bmatrix}
1
\end{bmatrix} \\
H &= I - 2 \frac{vv^T}{v^T v} = \begin{bmatrix}
-1
\end{bmatrix} \\
\text{ضرب } B \cdot H &= \begin{bmatrix}
-8.12 & 14.66 & -0 \\
 -0 & -1.96 & 0.5 \\
 -0 & -0 & 0
\end{bmatrix} \cdot \begin{bmatrix}
-1
\end{bmatrix} = \begin{bmatrix}
-8.12 & 14.66 & 0 \\
 -0 & -1.96 & -0.5 \\
 -0 & -0 & -0
\end{bmatrix} \\
\text{بروزرسانی } V &= \begin{bmatrix}
 1 & 0 & 0 \\
 0 & -0.65 & 0.76 \\
 0 & -0.76 & -0.65
\end{bmatrix}
\end{align*}

\subsection*{گام پنجم: بازتاب هوسهولدر از چپ \lr{(i = 2)}}
\begin{align*}
B &= \begin{bmatrix}
-8.12 & 14.66 & 0 \\
 -0 & -1.96 & -0.5 \\
 -0 & -0 & -0
\end{bmatrix} \\
x &= \begin{bmatrix}
-0
\end{bmatrix} \\
v &= \begin{bmatrix}
-1
\end{bmatrix} \\
H &= I - 2 \frac{vv^T}{v^T v} = \begin{bmatrix}
-1
\end{bmatrix} \\
\text{ضرب } H \cdot B &= \begin{bmatrix}
-1
\end{bmatrix} \cdot \begin{bmatrix}
-0 & -0 & -0
\end{bmatrix} = \begin{bmatrix}
-8.12 & 14.66 & 0 \\
 -0 & -1.96 & -0.5 \\
 0 & 0 & 0
\end{bmatrix} \\
\text{بروزرسانی } U &= \begin{bmatrix}
 0.41 \\
-0.82 \\
 0.41
\end{bmatrix} \cdot \begin{bmatrix}
-1
\end{bmatrix} = \begin{bmatrix}
-0.12 & 0.9 & -0.41 \\
-0.49 & 0.3 & 0.82 \\
-0.86 & -0.3 & -0.41
\end{bmatrix}
\end{align*}

در پایان، ماتریس \( B \) بیدیاگونال به صورت زیر است:
\[
B = \begin{bmatrix}
-8.12 & 14.66 & 0 \\
 -0 & -1.96 & -0.5 \\
 0 & 0 & 0
\end{bmatrix}
\]


