\section{مقدمه}


روش گولوب-کاهان یکی از الگوریتم‌های برجسته و کارآمد برای محاسبه تجزیه مقدار منفرد (SVD) است که توسط جین گولوب و ویلیام کاهان در سال 1965 معرفی شد. این روش به عنوان یک پیشرفت مهم در حوزه جبر خطی عددی شناخته می‌شود و به دلیل ویژگی‌های منحصر به فرد خود، به طور گسترده در علوم کامپیوتر، مهندسی و تحلیل داده‌ها مورد استفاده قرار می‌گیرد. الگوریتم گولوب-کاهان با بهره‌گیری از بازتاب‌های هوسهولدر، ماتریس اصلی را به یک ماتریس بیدیگونال تبدیل می‌کند. این تبدیل، فرآیند محاسبه SVD را بهینه‌سازی کرده و پیچیدگی محاسباتی را به طور قابل توجهی کاهش می‌دهد. 

یکی از مزایای برجسته این روش، پایداری عددی بالای آن است که باعث کاهش خطاهای گردشی و افزایش دقت نتایج می‌شود. بازتاب‌های هوسهولدر که در این الگوریتم به کار گرفته می‌شوند، امکان حفظ ساختار ماتریس‌های نادر (Sparse) را فراهم کرده و کارایی حافظه را بهبود می‌بخشند. این ویژگی‌ها، الگوریتم گولوب-کاهان را به یک انتخاب مناسب برای تحلیل داده‌های بزرگ‌مقیاس و کاربردهای علمی و مهندسی تبدیل کرده‌اند.

علاوه بر این، روش گولوب-کاهان با سایر روش‌های عددی، مانند روش‌های تکراری، سازگاری خوبی دارد و می‌تواند در ترکیب با آنها به نتایج بهتری منجر شود. در مقایسه با روش‌های مرسوم مانند الگوریتم ، گولوب-کاهان نه تنها از لحاظ کارایی و دقت برتری دارد، بلکه از نظر پایداری عددی نیز مطمئن‌تر است. این ویژگی‌ها باعث شده‌اند که این روش در بسیاری از زمینه‌های کاربردی، از جمله پردازش تصویر، تحلیل داده‌های ژنومی و مدل‌سازی‌های علمی، مورد استفاده قرار گیرد.

در نتیجه، الگوریتم گولوب-کاهان به عنوان یکی از ابزارهای اساسی در جبر خطی عددی شناخته می‌شود که توانسته است با بهبود کارایی محاسبات و حفظ دقت نتایج، سهم بسزایی در پیشرفت علوم محاسباتی و تحلیل داده‌ها ایفا کند. این روش همچنان به عنوان یک استاندارد در بسیاری از کاربردهای عملی و تحقیقاتی مورد توجه و استفاده قرار دارد.
