\section{الگوریتم گولوب-کاهان}
الگوریتم گولوب-کاهان به دو مرحله اصلی تقسیم می‌شود:

\begin{itemize}
    \item تبدیل ماتریس به ماتریس بیدیگونال 

    \item و محاسبه SVD از شکل بیدیگونال.

\end{itemize}






این مراحل در زیر با جزئیات توضیح داده شده‌اند.

\subsection{تبدیل ماتریس به ماتریس بیدیگونال}

\subsubsection{ماتریس بیدیگونال}


ماتریس بیدیگونال \lr{(Bidiagonal Matrix)} یک ماتریس مربعی است که تنها دارای عناصر غیر صفر بر روی قطر اصلی و یک باند زیر یا بالای قطر اصلی است. به عبارتی، در ماتریس بیدیگونال فقط عناصر روی قطر اصلی و عناصر یک ردیف یا یک ستون بالاتر یا پایین‌تر از قطر اصلی غیر صفر هستند.


در ماتریس‌ها، اصطلاحات "باند بالا" و "باند پایین" به عناصر غیرصفر ماتریس در نزدیکی قطر اصلی اشاره دارند. بیایید به تعریف و مثال‌های بیشتری بپردازیم تا مفهوم آن را روشن‌تر کنیم.

\paragraph{ماتریس بیدیگونال بالا}



یک ماتریس بیدیگونال بالا به شکل زیر است:


\[
B_{\text{up}} = \begin{pmatrix}
a_{11} & a_{12} & 0 & 0 \\
0 & a_{22} & a_{23} & 0 \\
0 & 0 & a_{33} & a_{34} \\
0 & 0 & 0 & a_{44}
\end{pmatrix}
\]

که در آن \(a_{11}, a_{12}, a_{22}, a_{23}, a_{33}, a_{34}, a_{44}\) می‌توانند هر مقدار عددی باشند.

مثال عددی بیدیگونال بالا:

\[
B_{\text{up}} = \begin{pmatrix}
4 & 2 & 0 & 0 \\
0 & 3 & 1 & 0 \\
0 & 0 & 5 & 3 \\
0 & 0 & 0 & 6
\end{pmatrix}
\]

در این مثال، عناصر غیرصفر در باند بالای قطر اصلی عبارتند از \(2, 1, 3\).


\paragraph{ماتریس بیدیگونال پایین}


یک ماتریس بیدیگونال پایین به شکل زیر است:

\[
B_{\text{low}} = \begin{pmatrix}
a_{11} & 0 & 0 & 0 \\
a_{21} & a_{22} & 0 & 0 \\
0 & a_{32} & a_{33} & 0 \\
0 & 0 & a_{43} & a_{44}
\end{pmatrix}
\]

که در آن \(a_{11}, a_{21}, a_{22}, a_{32}, a_{33}, a_{43}, a_{44}\) می‌توانند هر مقدار عددی باشند.

مثال عددی بیدیگونال پایین:

\[
B_{\text{low}} = \begin{pmatrix}
4 & 0 & 0 & 0 \\
2 & 3 & 0 & 0 \\
0 & 1 & 5 & 0 \\
0 & 0 & 3 & 6
\end{pmatrix}
\]

در این مثال، عناصر غیرصفر در باند پایین قطر اصلی عبارتند از \(2, 1, 3\).





در الگوریتم گولوپ-کاهان برای محاسبه SVD، معمولاً از ماتریس بیدیگونال بالا \lr{(upper bidiagonal matrix)} استفاده می‌شود. این به این معناست که ماتریس اولیه به صورتی تبدیل می‌شود که عناصر غیرصفر تنها در قطر اصلی و یک ردیف بالاتر از قطر اصلی قرار دارند. این نوع ماتریس بیدیگونال بالا به دلیل ساختار ساده‌تری که دارد، برای محاسبات عددی و پیاده‌سازی الگوریتم‌های تجزیه مانند SVD مناسب‌تر است.

\pagebreak

\subsection{الگوریتم بیدیاگونالیزاسیون گولوب-کاهان}

\subsection*{ورودی و خروجی‌ها}
\begin{itemize}
  \item \textbf{ورودی}: ماتریس \( A \) با ابعاد \( m \times n \)
  \item \textbf{خروجی‌ها}:
    \begin{itemize}
      \item ماتریس بیدیاگونال \( B \) با ابعاد \( m \times n \)
      \item ماتریس ارتوگونال \( U \) با ابعاد \( m \times m \)
      \item ماتریس ارتوگونال \( V \) با ابعاد \( n \times n \)
    \end{itemize}
\end{itemize}

\subsection*{مقداردهی اولیه}
\begin{align*}
  U &= I_m \quad \text{(ماتریس واحد \( m \times m \))} \\
  V &= I_n \quad \text{(ماتریس واحد \( n \times n \))} \\
  B &= A
\end{align*}

\subsection*{حلقه اصلی}
برای \( i = 1 \) تا \( \min(m, n) \):


 استفاده از \lr{`min(m, n)`} کمک می‌کند تا عملیات bidiagonalization به صورت بهینه‌تری انجام شود، زیرا ما تنها تا حداقل اندازه کوچکترین بُعد از ماتریس A نیاز داریم. این کمک می‌کند که الگوریتم به سرعت به نقاط کلیدی از فرایند برسد.

 
\begin{enumerate}
  \item \textbf{بازتاب هوسهولدر از چپ}
  بازتاب هوسهولدر از چپ به این صورت عمل می‌کند که با اعمال ماتریس هوسهولدر از      سمت چپ به یک ماتریس، مولفه‌های زیر قطر اصلی در یک ستون خاص صفر می‌شوند. به طور مشخص، اگر بازتاب هوسهولدر به یک ماتریس \( A \) از سمت چپ اعمال شود، عناصر زیر قطر اصلی در یک ستون خاص از آن ماتریس صفر خواهند شد.


  
  \begin{itemize}
    \item بردار ستون \( x \) از ماتریس \( B \) تعریف می‌شود به صورت:
    \[
    x = B[i:m, i]
    \]
این نشان می‌دهد که \( x \) برابر است با بردار ستونی از ماتریس \( B \) که از ردیف \( i \)ام تا ردیف \( m \)ام و در ستون \( i \)ام قرار دارد.
    
    \item بردار بازتاب هوسهولدر \( v \) برای \( x \) محاسبه می‌شود:
    \[
    v = x + \text{sign}(x_1) \|x\|_2 e_1
    \]
    که \( e_1 \) بردار واحد اول است و \(\text{sign}(x_1)\) علامت اولین مولفه \( x \) است.
    \item ماتریس هوسهولدر \( H \) ساخته می‌شود:
    \[
    H = I_{m-i} - 2 \frac{vv^T}{v^T v}
    \]

    در الگوریتم بازتاب هوسهولدر از سمت چپ، ماتریس \( H \) یک ماتریس مربعی است با ابعاد \( (m-i) \times (m-i) \). 
    \item ماتریس \( B \) به‌روزرسانی می‌شود:
    \[
    B[i:m, i:n] = H B[i:m, i:n]
    \]

    زیرماتریس \( B[i:m, i:n] \) ابعاد \( (m-i) \times (n-i) \) دارد.

    
     \textbf{زیرماتریس \( B \)}:
    \begin{itemize}
        \item عبارت \( B[i:m, i:n] \) به زیرماتریسی از \( B \) اشاره دارد که از ردیف \( i \) تا ردیف \( m \) و از ستون \( i \) تا ستون \( n \) قرار دارد. این زیرماتریس شامل المان‌هایی است که تحت تاثیر بازتاب هوسهولدر قرار می‌گیرند.
    \end{itemize}
    \item ماتریس \( U \) به‌روزرسانی می‌شود:
    \[
    U[:, i:m] = U[:, i:m] H
    \]

     \textbf{زیرماتریس \( U \)}:
    \begin{itemize}
        \item عبارت \( U[:, i:m] \) به زیرماتریسی از \( U \) اشاره دارد که شامل تمام ردیف‌ها و ستون‌های \( i \) تا \( m \) است. این زیرماتریس نشان‌دهنده بخشی از \( U \) است که باید تحت تاثیر ماتریس هوسهولدر \( H \) قرار گیرد.

           ابعاد  \( (m) \times (m-i) \)   
    \end{itemize}

  \end{itemize}

  \item \textbf{بازتاب هوسهولدر از راست}
  زتاب هوسهولدر از راست باعث صفر شدن درایه‌های زیر قطر اصلی ستون‌های ماتریس می‌شود، در حالی که بازتاب هوسهولدر از سمت چپ باعث صفر شدن درایه‌های بالای قطر اصلی سطرهای ماتریس می‌شود.
  \begin{itemize}
    \item اگر \( i < n  \):
    \begin{itemize}
      \item بردار ردیف \( x \) از ماتریس \( B \) تعریف می‌شود به صورت:
      \[
      x = B[i, i+1:n]
      \]
      ، \( x \) یک بردار است که از سطر \( i \) ام ماتریس \( B \) استخراج شده است، از ستون \( i+1 \) تا ستون \( n \).


      \item بردار بازتاب هوسهولدر \( v \) برای \( x \) محاسبه می‌شود:
      \[
      v = x + \text{sign}(x_1) \|x\|_2 e_1
      \]
      \item ماتریس هوسهولدر \( H \) ساخته می‌شود:
      \[
      H = I_{n-i-1} - 2 \frac{vv^T}{v^T v}
      \]
      \item ماتریس \( B \) به‌روزرسانی می‌شود:
      \[
      B[i:m, i+1:n] = B[i:m, i+1:n] H
      \]

     \textbf{زیرماتریس \( B \)}:

       این عبارت به معنی این است که بخشی از ماتریس \( B \) که از سطر \( i \) تا سطر \( m-1 \) و از ستون \( i+1 \) تا ستون \( n-1 \) است، 
      \item ماتریس \( V \) به‌روزرسانی می‌شود:
      \[
      V[:, i+1:n] = V[:, i+1:n] H
      \]
      \pagebreak
    \end{itemize}
  \end{itemize}



\subsection{تجزیه SVD  به وسیله ماتریس بیدیگونال }

ماتریس بیدیگونال \( B \) می‌تواند به صورت زیر تجزیه شود:
\[
B = U_b \Sigma (V_b)^T
\]
که در اینجا:
\begin{itemize}
    \item \( U_b \) و \( V_b \) ماتریس‌های ارتوگونال برای بردارهای ویژه چپ و راست هستند.
    \item \( \Sigma \) بردار وکتور ویژه‌های ماتریس بیدیگونال \( B \) است.
\end{itemize}

\section*{2. به‌روزرسانی ماتریس‌های ارتوگونال \( U \) و \( V \):}
ماتریس‌های ارتوگونال \( U \) و \( V \) بر اساس بردارهای ویژه به‌دست آمده از SVD به‌روزرسانی می‌شوند:
\[
U = U U_b
\]
\[
V = V V_b^T
\]
که این عملیات باعث به‌روزرسانی ماتریس‌های ارتوگونال \( U \) و \( V \) می‌شود.

در واقغ ماتریس \( B \) که  از ماتریس \( A \)   بیدیگونال شده است پس از تجزیه SVD و به روز رسانی ماتریس های  \( U_b \) و \( V_b \)  و به دست اوردن ماتریس های  \( U \) و \( V \) میتوان راطه زیر را نوشنت .


\[
A = U \Sigma (V)^T
\]


\pagebreak

\end{enumerate}


